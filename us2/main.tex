\documentclass{beamer}

\usepackage{beamerthemesplit}
% \usepackage[orientation=landscape,size=custom,width=16,height=11,scale=0.4,debug]{beamerposter} 
\usepackage[utf8]{inputenc}
\usepackage[english]{babel}
\usepackage{ulem}

\usepackage{minted}
\usemintedstyle{borland}

\usepackage{amsmath}
\usepackage{multicol}
\usepackage{graphicx} %Loading the package
\graphicspath{{../img/}} %Setting the path

\usetheme{Madrid}
\usecolortheme{default}

%------------------------------------------------------------
%This block of code defines the information to appear in the
%Title page
\title[Progress Report] %optional
{Progress Report}

\subtitle{Lifting Linearization of a UAV}

\author[Sarmiento Fernando] % (optional)
{Sarmiento Diaz Fernando Gabriel}

\institute[TITECH] % (optional)
{
  \inst{1}%
  Yamakita Laboratory\\
  Department of Systems and Control Engineering\\
  Tokyo Institute of Technology
}

\date[2021] % (optional)
\today
% {Very Large Conference, April 2021}

% \logo{\includegraphics[height=1cm]{overleaf-logo}}

%End of title page configuration block
%------------------------------------------------------------



%------------------------------------------------------------
%The next block of commands puts the table of contents at the 
%beginning of each section and highlights the current section:

\AtBeginSection[]
{
  \begin{frame}
    \frametitle{Table of Contents}
    \tableofcontents[currentsection]
  \end{frame}
}
%------------------------------------------------------------


\begin{document}

%The next statement creates the title page.
\frame{\titlepage}


%---------------------------------------------------------
%This block of code is for the table of contents after
%the title page
\begin{frame}
    \frametitle{Table of Contents}
    \tableofcontents
\end{frame}
%---------------------------------------------------------

\section{Linear Models for Regression}
\subsection{The Evidence Approximation}
\begin{frame}
    \frametitle{The Evidence Approximation(1)}
    The fully Bayesian predictive distribution is given
    \begin{equation}
        p(t \mid \mathbf{t})=\iiint p(t \mid \mathbf{w}, \beta) p(\mathbf{w} \mid \mathbf{t}, \alpha, \beta) p(\alpha, \beta \mid \mathbf{t}) \mathrm{d} \mathbf{w} \mathrm{d} \alpha \mathrm{d} \beta
    \end{equation}
    but this integral is intractable. Approximate with
    \begin{equation}
        p(t \mid \mathbf{t}) \simeq p(t \mid \mathbf{t}, \widehat{\alpha}, \widehat{\beta})=\int p(t \mid \mathbf{w}, \widehat{\beta}) p(\mathbf{w} \mid \mathbf{t}, \widehat{\alpha}, \widehat{\beta}) \mathrm{d} \mathbf{w}
    \end{equation}

    where $(\widehat{\alpha}, \widehat{\beta})$ is the mode of $p(\alpha, \beta \mid \mathbf{t})$, which is assumed to be sharply peaked; a.k.a. \textit{empirical Bayes}, \textit{type II} or \textit{generalized maximum likelihood}, or \textit{evidence approximation}
    
\end{frame}
\begin{frame}
    \frametitle{The Evidence Approximation(2)}
    From Bayes' theorem we have
    \begin{equation}
        p(\alpha, \beta \mid \mathbf{t}) \propto p(\mathbf{t} \mid \alpha, \beta) p(\alpha, \beta)
    \end{equation}
    and if we assume $ p(\alpha, \beta)$ to be flat we see that
    \begin{equation}
        \begin{aligned}
            p(\alpha, \beta \mid \mathbf{t}) & \propto p(\mathbf{t} \mid \alpha, \beta)                                                     \\
                                             & =\int p(\mathbf{t} \mid \mathbf{w}, \beta) p(\mathbf{w} \mid \alpha) \mathrm{d} \mathbf{w} .
        \end{aligned}
    \end{equation}
    General results for Gaussian integrals give
    \begin{equation}
        \ln p(\mathbf{t} \mid \alpha, \beta)=\frac{M}{2} \ln \alpha+\frac{N}{2} \ln \beta-E\left(\mathbf{m}_{N}\right)+\frac{1}{2} \ln \left|\mathbf{S}_{N}\right|-\frac{N}{2} \ln (2 \pi)
    \end{equation}
\end{frame}
%---------------------------------------------------------
\end{document}